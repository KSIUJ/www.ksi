\documentclass{article}
\usepackage[polish]{babel}
\usepackage[utf8]{inputenc} 
\usepackage{polski} 
\usepackage{enumerate}
\frenchspacing 	
\usepackage{indentfirst}
\usepackage[total={180mm,297mm},top=20mm, bottom=20mm, left=15mm, includefoot, foot=15mm]{geometry}

\title{Statut Koła Studentów Informatyki Uniwersytetu Jagiellońskiego} 
\date{23 października 2015}

\begin{document}
\maketitle

\vspace{0.2cm}
\begin{center}
    \section*{Rozdział 1: Postanowienia ogólne}
\end{center}\vspace{0.4cm}
	\paragraph{\S 1. Organizacja nosi nazwę: Koło Studentów Informatyki Uniwersytetu Jagiellońskiego.}
	\paragraph{\S 2. Siedziba Koła jest pokój 1173 w budynku Wydziału Matematyki i Informatyki przy
\mbox{ul. Łojasiewicza~6.}}
	\paragraph{\S 3. Opiekunem Koła jest pracownik Instytutu Informatyki i Matematyki Komputerowej Wydziału Matematyki i Informatyki UJ
         wyznaczony przez dyrektora powyższej jednostki i zaaprobowany przez Zarząd Koła.
        }
	\paragraph{\S 4. Źródłami prawa obowiązującego członków Koła są: Statut Koła, uchwały Walnych Zebrań, rozporządzenia Zarządu Koła oraz
         powszechnie obowiązujące prawo Rzeczypospolitej Polskiej.
        }

\begin{center}\vspace{0.4cm}
    \section*{Rozdział 2: Cele i Zadania}
\end{center}\vspace{0.4cm}
	\paragraph{\S 1. Celami Koła są:}
		\begin{enumerate}
			\item uzupełnianie i pogłębianie wiedzy i umiejętności studentów ze szczególnym uwzględnieniem nauk związanych z informatyką;
			\item rozbudzanie wśród studentów zainteresowań naukowych i pasji badawczych;
			\item promowanie osiągnięć naukowych członków Koła;
			\item przygotowanie studentów do przyszłej pracy naukowej i zawodowej;
			\item rozwijanie stosunków koleżeńskich i towarzyskich między członkami Koła;
			\item propagowanie wiedzy informatycznej;
			\item poszerzanie horyzontów i zainteresowań studentów;
		\end{enumerate}
	

	\paragraph{\S 2. Sposobami prowadzącymi do tych celów są:}
\begin{enumerate}
			\item urządzanie odczytów, referatów, zebrań dyskusyjnych, prelekcji;
			\item wydawanie skryptów i rozpraw naukowych;
			\item urządzanie wycieczek i obozów naukowych;
			\item udostępnianie posiadanego księgozbioru i czasopism specjalistycznych;
			\item pośrednictwo w zakupie książek i pomocy naukowych;
			\item współpraca z innymi organizacjami o podobnych celach;
			\item występowanie wobec władz w interesach Koła i sprawach jego członków;
			\item organizowanie spotkań, zabaw i innych imprez towarzyskich.
		\end{enumerate}

\vspace{0.4cm}
\begin{center}
    \section*{Rozdział 3: Członkowie Koła}
\end{center}\vspace{0.4cm}
\paragraph{\S 1. Członkowie Koła dzielą się na:}
\begin{enumerate}
			\item zwyczajnych;
			\item sympatyków;
			\item honorowych.
		\end{enumerate}
\paragraph{\S 2. Członkiem zwyczajnym może być każdy student UJ. }
\paragraph{\S 3. Członkiem sympatykiem może być dowolna osoba za zgodą zarządu.}
\paragraph{\S 4. Członek zwyczajny kola, w momencie ukończenia studiów lub rezygnacji z dalszego studiowania, pozostaje członkiem kola z prawami członka sympatyka. }
\paragraph{\S 5. Godność członka honorowego nadaje Walne Zebranie na wniosek Zarządu osobom szczególnie zasłużonym dla Koła.}
\paragraph{\S 6. Zarząd Koła może osobie ubiegającej się odmówić przyjęcia w poczet członków Koła, a na
wniosek nieprzyjętego zobowiązany jest do wydania pisemnego uzasadnienia tej decyzji.}
\paragraph{\S 7. Członkowie zwyczajni maja prawo:}
\begin{enumerate}
			\item glosowania na Walnym Zebraniu;
			\item czynnego i biernego wyboru do Władz Koła;
			\item brania udziału we wszystkich zebraniach Koła;
			\item stawiania wniosków i zgłaszania projektów uchwal do rozpatrzenia na Walnym Zebraniu;
			\item organizowania pod patronatem Koła i za zgoda Zarządu działalności wyszczególnionej w §2 Rozdziału 2;
			\item brania udziału we wszystkich imprezach i wyjazdach organizowanych przez Koło;
			\item wprowadzania gości na imprezy Koła za zgoda Zarządu;
			\item korzystania z wszelkich dostępnych pomocy naukowych będących własnością Koła;
			\item korzystania z możliwości rozwoju dawanych im przez Kolo;
			\item odwołania się od decyzji Zarządu do Komisji Rewizyjnej;
		\end{enumerate}
\paragraph{\S 8. Członkowie sympatycy maja prawa analogiczne jak w §7 Rozdziału 3, z wyjątkiem pkt. 1, 2, 8 a ponadto maja prawo:}
\begin{enumerate}
			\item brania udziału we wszystkich zebraniach Koła z głosem doradczym;
			\item korzystania z biblioteki i urządzeń Koła w lokalu Koła;
		\end{enumerate}
\paragraph{\S 9. Członkowie honorowi maja prawa analogiczne jak w §7 Rozdziału 3, z wyjątkiem pkt. 1, 2, 8, sa zwolnieni z opłat składki członkowskiej,
           a ponadto maja prawo do korzystania z biblioteki i urządzeń Koła w lokalu Koła oraz brania udziału we wszystkich zebraniach Koła z głosem doradczym.}
\paragraph{\S 10. Z wszelkich zasobów Koła mogą korzystać członkowie Koła i osoby przez nie zaproszone. }
\paragraph{\S 11. Obowiązki członków Koła:}
\begin{enumerate}
			\item wywiązywanie się z podjętych względem Koła zobowiązań;
			\item przestrzeganie przepisów i zarządzeń Władz Koła;
			\item regularne opłacanie składki członkowskiej w terminie do momentu rozpoczęcia Zwyczajnego Walnego Zebrania w danym roku akademickim;
			\item dbanie o dobre imię oraz mienie Koła;
			\item utrzymywanie czystości w siedzibie Koła;
			\item obecność na Walnym Zebraniu;
			\item stosowanie sie do Statutu i uchwal Walnego Zebrania;
			\item w miarę możliwości, dzielenie się własna wiedza i umiejętnościami z innymi członkami Koła.
		\end{enumerate}
\paragraph{\S 12. Czasowe zawieszenie w prawach członka Koła nastepuje:}
\begin{enumerate}
			\item wskutek decyzji Zarządu z podaniem pisemnego uzasadnienia oraz terminu ustania zawieszenia;
			\item automatycznie wskutek nieopłacenia aktualnej składki członkowskiej i trwa do czasu uregulowania zaległych płatności.
		\end{enumerate}
\paragraph{\S 13. Członkostwo Koła ustaje wskutek:}
		\begin{enumerate}
			\item pisemnego zgłoszenia do Zarządu Koła swojej rezygnacji;
			\item wykluczenia przez Zarząd z podaniem pisemnego uzasadnienia decyzji;
			\item w przypadku czlonków zwyczajnych i sympatyków – decyzji Zarzadu Koła spowodowanej nie oplaceniem skladki czlonkowskiej
			zgodnie z pkt. 3 §11 Rozdziału 3.
			\item nie opłacania składek członkowskich przez okres dłuższy niż rok (nie jest wymagana decyzja Zarządu Koła). 
		\end{enumerate}

\vspace{0.4cm}
\begin{center}
    \section*{Rozdział 4: Władze Koła}
\end{center}\vspace{0.4cm}
		\paragraph{\S 1. Władzami Koła sa:}
		\begin{enumerate}
			\item Walne Zebranie;
			\item Zarząd Koła;
			\item Komisja Rewizyjna.
		\end{enumerate}
		\paragraph{\S 2. Walne Zebranie jest najwyższa władza Koła.}
		\paragraph{\S 3. Walne Zebranie może byc:}
		\begin{enumerate}
			\item zwyczajne;
			\item nadzwyczajne.
		\end{enumerate}
		\paragraph{\S 4. Zwyczajne Walne Zebranie powołuje Zarząd Koła w październiku każdego roku.}
		\paragraph{\S 5. Nadzwyczajne Walne Zebranie zwoluje:}
		\begin{enumerate}
			\item Zarząd
		\begin{enumerate}[(a)]
			\item z własnej inicjatywy;
			\item na pisemne zarządzenie Komisji Rewizyjnej z podaniem porządku obrad, w terminie do 14 dni od złożenia zarządzenia;
			\item na wniosek 1/4 zwyczajnych członków Koła złożony na piśmie z podaniem porządku obrad, w terminie do 14 dni od złożenia wniosku;
			\item na wniosek opiekuna Koła złożony na piśmie z podaniem porządku obrad, w terminie 14 dni od złożenia wniosku.
		\end{enumerate}
			\item Komisja Rewizyjna w przypadku bezczynności Zarządu w pkt.1. §5 Rozdziału 4.
			\item Opiekun Koła w przypadku bezczynności Komisji Rewizyjnej w pkt. 2. §5 Rozdziału 4.
		\end{enumerate}
		\paragraph{\S 6. Termin pierwszy i drugi, miejsce oraz początek obrad Walnego Zebrania powinny być ogłoszone na tablicach Koła,
                           w lokalu Koła z wyprzedzeniem co najmniej siedmiodniowym. W miarę możliwości informacje te powinny zostać ogłoszone także droga elektroniczna.}
		\paragraph{\S 7. Do odbycia Walnego Zebrania w pierwszym terminie potrzebne jest od chwili rozpoczęcia obecność więcej niż polowy zwyczajnych członków Koła.}
		\paragraph{\S 8. W przypadku nie odbycia się Walnego Zebrania w pierwszym terminie z powodu braku wymaganej ilości osób,
                           Walne Zebranie odbyte w drugim terminie jest prawomocne bez względu na ilość obecnych członków Koła.}
		\paragraph{\S 9. Uchwały Walnego Zebrania zapadają zwyczajna większością głosów z wyjątkiem uchwały o rozwiązaniu Koła (Rozdział 4).}
		\paragraph{\S 10. Do kompetencji Walnego Zebrania naleza:}
		\begin{enumerate}
			\item wybór i odwoływanie członków Zarządu i Komisji Rewizyjnej;
			\item przyjmowanie lub odrzucanie sprawozdań ustępującego Zarządu i Komisji Rewizyjnej;
			\item udzielanie, względnie nie udzielania absolutorium ustępującemu Zarządowi;
			\item zmiana statutu Koła;
			\item rozpatrywanie wniosków oraz interpelacji Komisji Rewizyjnej i członków Koła;
			\item rozpatrywanie odwołań od decyzji Zarządu;
			\item nadawanie godności członka honorowego;
			\item podejmowanie wszelkich decyzji leżących w kompetencjach Zarządu lub Komisji Rewizyjnej, o ile Walne Zebranie uzna to za konieczne.
		\end{enumerate}
		\paragraph{\S 11. W razie odrzucenia wniosku Komisji Rewizyjnej o udzielenie absolutorium ustępującemu Zarządowi,
                           Walne Zebranie wybiera komisje złożona z trzech osób nie wchodzących w skład Zarządu ani tez Komisji Rewizyjnej, która w
                           przeciągu 14 dni roboczych zbada całokształt działalności Zarządu i niezwłocznie przedłoży Nadzwyczajnemu Walnemu Zebraniu
                           wnioski o wyniku swojej pracy.}
		\paragraph{\S 12. Obradami Walnego Zebrania kieruje wybierany każdorazowo przewodniczący.}
		\paragraph{\S 13. Zarząd Koła składa się z Prezesa, dwóch Zastępców Prezesa, Skarbnika i Sekretarza.}
		\paragraph{\S 14. Kadencja Zarządu trwa od momentu wyboru do najbliższego Zwyczajnego Walnego Zebrania.}
		\paragraph{\S 15. Uchwały Zarządu zapadają zwykła większością głosów, przy obecności więcej niż połowy
                           członków Zarządu.}
		\paragraph{\S 16. W glosowaniach nierozstrzygniętych decyduje Prezes.}
		\paragraph{\S 17. Do kompetencji kolegialnych Zarządu naleza:}
		\begin{enumerate}
			\item zwoływanie Walnych Zebrań;
			\item zapewnienie wykonania uchwal Walnego Zebrania;
			\item zatwierdzenie sprawozdań członków Zarządu;
			\item zatwierdzenie preliminarza finansowego na najbliższy rok;
			\item ustalenie wysokości wpisowego i składek;
			\item podpisywanie wszelkich pism i zarządzeń;
			\item dysponowanie majątkiem Koła w ramach ustaleń Walnego Zebrania;
			\item tworzenie i rozwiązywanie sekcji według reguł ustalanych przez siebie;
			\item wykluczenie członków;
			\item zawieszenie w czynnościach członków Zarządu z wyjątkiem Prezesa;
			\item załatwienie bieżących spraw Koła nie mieszczących się w kompetencjach żadnego z członków Zarządu;
			\item przyjmowanie składek członkowskich.
			\item powoływanie i odwoływanie Gospodarza(rzy) Koła oraz Administratora(ów) serwerowni Koła.
		\end{enumerate}
		\paragraph{\S 18. Do kompetencji Prezesa naleza:}
		\begin{enumerate}
			\item reprezentowanie Koła na zewnątrz;
			\item promowanie Koła i dbanie o aktualność strony internetowej Koła;
			\item zwoływanie i prowadzenie zebrań Zarządu Koła;
			\item kontrolowanie pracy członków Zarządu;
			\item pisanie i przedkładanie Zarządowi i Walnemu Zebraniu rocznego sprawozdania z działalności Koła.
		\end{enumerate}
		\paragraph{\S 19. Kompetencje Zastępców Prezesa ustala Prezes w ramach swego własnego zakresu czynności, na pierwszym zebraniu Zarządu.}
		\paragraph{\S 20. Do obowiązków Sekretarza naleza:}
		\begin{enumerate}
			\item Wyznaczanie reprezentanta na zebrania organizacji zrzeszających kola naukowe UJ, do których Kolo należy.
			\item prowadzenie korespondencji Koła;
			\item protokołowanie wszelkich zebrań organizowanych przez Kolo;
			\item opieka nad pieczęcią Koła.
			\item utrzymywanie porządku w aktualnej dokumentacji Koła oraz w archiwum.
		\end{enumerate}
		\paragraph{\S 21. Do obowiązków Skarbnika nalezy:}
		\begin{enumerate}
			\item bezpośrednie zarządzanie funduszami Koła według dyspozycji Zarządu;
			\item prowadzenie książki kasowej;
			\item prowadzenie ewidencji członków Koła i wpisywanie nowych członków;
			\item rejestrowanie składek członkowskich oraz dokonywanie wszelkich innych operacji finansowych;
			\item przedstawienie Zarządowi Koła i Walnemu Zebraniu rocznego sprawozdania finansowego;
			\item sporządzanie projektu preliminarza rocznego i przedkładanie go Zarządowi Koła do zatwierdzenia;
		\end{enumerate}
		\paragraph{\S 22. Gospodarz Koła jest zobowiązany do:}
		\begin{enumerate}
			\item sprawowania pieczy nad majątkiem trwałym i ruchomym Koła;
			\item udostępniania sprzętu, książek itp. rzeczy należących do Koła;
			\item prowadzenia ksiąg inwentarzowych;
			\item utrzymywania czystości i porządku w lokalu Koła;
			\item dbanie o przyrządu kancelaryjne;
			\item publikowania plakatów aktualnych imprez organizowanych przez Kolo;
			\item dbania o czystość tablic informacyjnych Koła oraz o aktualność znajdujących na nich ogłoszeń i plakatów.
		\end{enumerate}
		\paragraph{\S 23.  W przypadku rezygnacji lub trwałej niemożności sprawowania funkcji przez członka Zarządu
                           Koła, Zarząd Koła jest zobowiązany do wyznaczenia osoby z Zarządu Koła, która przejmie
                           obowiązki rezygnującego. W przypadku niemożności dokonania decyzji Zarząd Koła jest
                           zobowiązany do zwołania Nadzwyczajnego Walnego Zebrania w terminie dwóch tygodni od
                           rezygnacji, w celu wybrania nowego Zarządu Koła.
                }
		\paragraph{\S 24. Administrator serwerowni Koła jest odpowiedzialny za:}
		\begin{enumerate}
			\item sprawne działanie urządzeń w serwerowni;
			\item mienie Koła znajdujące się w serwerowni;
			\item porządek w serwerowni;
		\end{enumerate}
		\paragraph{\S 25. Administrator serwerowni Koła wraz z Zarządem Koła ustala regulamin korzystania z serwerów i innego sprzętu elektronicznego
                           wchodzącego w skład zasobów Koła.}
		\paragraph{\S 26. Komisja Rewizyjna jest organem kontrolnym finansów i działalności Zarządu Koła.}
		\paragraph{\S 27. Komisja Rewizyjna składa się z dwóch członków nie wchodzących w skład Zarządu Koła i wybieranych na okres miedzy dwoma kolejnymi
                           Zwyczajnymi Walnym Zebraniami.}
		\paragraph{\S 28. Do obowiązków Komisji Rewizyjnej nalezy:}
		\begin{enumerate}
			\item kontrola czynności administracyjnych Zarządu Koła;
			\item kontrola gospodarki finansowej Zarządu Koła;
			\item czuwanie nad ścisłym przestrzeganiem i realizowaniem postanowień Walnego Zebrania;
			\item składanie Walnemu Zebraniu sprawozdania ze swej działalności.
		\end{enumerate}
		\paragraph{\S 29. Komisja Rewizyjna ma prawo:}
		\begin{enumerate}
			\item składać Zarządowi wniosek o zwołanie Nadzwyczajnego Walnego Zebrania zgodnie z §5 Rozdziału 4 pkt 1 b;
			\item zawiesić postanowienie Zarządu sprzeczne z postanowieniami Walnego Zebrania;
			\item przeprowadzić kontrole działalności Koła
			\begin{enumerate}[(a)]
				\item z własnej inicjatywy;
				\item na pisemne zadanie członka Zarządu Koła;
				\item na trzy dni przed Walnym Zebraniem.
			\end{enumerate}
			\item zwołać Nadzwyczajne Walne Zebranie według §5 Rozdziału 4 pkt 2.
		\end{enumerate}
		\paragraph{\S 30. Komisja Rewizyjna ma obowiązek przeprowadzać kontrole działalności Koła}
		\begin{enumerate}
			\item na pisemne zadanie Zarządu Koła.
			\item na pisemne zadanie zwyczajnych członków Koła.
		\end{enumerate}
        \paragraph{\S 31. W przypadku rezygnacji lub trwalej niemożności sprawowania funkcji przez członka Komisji Rewizyjnej, pozostali
                   członkowie Komisji Rewizyjnej lub Zarząd Koła maja obowiązek zwołania Nadzwyczajnego Walnego Zebrania w terminie dwóch tygodni od
                   rezygnacji członka Komisji Rewizyjnej, w celu przeprowadzenia wyborów uzupełniających do Komisji Rewizyjnej.}
		\paragraph{\S 32. Członkiem Zarządu Koła może być wyłącznie osoba będąca członkiem zwyczajnym Koła od co najmniej 6 miesięcy w dniu wyboru.}
		\paragraph{\S 33. W szczególnych przypadkach za zgodą walnego zebrania członkiem zarządu koła może zostać dowolny członek zwyczajny zgłoszony przez osobę spełniającą warunki \S 32.}
\vspace{0.4cm}
\begin{center}
    \section*{Rozdział 5: Rozwiązanie Koła:}
\end{center}\vspace{0.4cm}
		\paragraph{\S 1. Rozwiązanie Koła może nastąpić na mocy uchwały Walnego Zebrania odbytego w obecności 3/4 liczby zwyczajnych członków Koła
                           i większością 3/4 głosów, za wiedza i zgoda Opiekuna Koła.}
		\paragraph{\S 2. Czynności likwidacyjne wykonują likwidatorzy powołani przez Walne Zebranie.}
		\paragraph{\S 3. Pozostali po Kole majątkiem rozporządza Walne Zebranie.}
\end{document}
